\chapter*{Prologue} %Exordium, Prelude, Introduction

\section*{Introduction}

Software development has changed a lot over the years, moving from manual steps to automated workflows.
Old methods had long manual separate steps, often causing problems.
To deliver faster and improve quality modern development has shifted to automation.
Machine Learning development also moves towards more automation\cite{Haakman2021}.

According to\cite{10.1145/3533378}, only a small percentage of machine learning projects successfully reach production deployment, due to issues and operational complexity\cite{Haakman2021}.
The emergence of MLOps as a distinct discipline represents a response to these challenges, offering structured approaches to model development, deployment, monitoring, and governance\cite{treveil2020introducing}.
This thesis begins with a comprehensive state-of-the-art review of MLOps methodologies, frameworks, and tools currently employed across industry and research domains.
Our review is motivated by the need to establish a solid theoretical and practical foundation for developing a flexible MLOps workflow that can address the diverse requirements of modern enterprises and research organizations.
We aim to identify in the literature the core components and principles that constitute effective MLOps implementations\cite{DBLP:journals/corr/abs-2103-08942}.

The insights gained from this review directly inform our subsequent implementation project, where we develop a modular, adaptable MLOps workflow designed to accommodate various scales of machine learning projects.
Our implementation strategy follows a two-phase approach: first, we develop and validate our workflow using a controlled demonstration model that shows core functionality in a simplified context.
This allows us to establish baseline capabilities and verify fundamental workflow mechanics without the complexities of a production system.
Following this initial validation, we demonstrate the workflow's flexibility and robustness by applying it to a real-world production system—the LSFB (Langue des Signes de Belgique Francophone) model developed at our university\cite{9534336}.
By adapting our workflow to enhance this existing system, we provide tangible evidence of its practical utility and adaptability.

\section{Acknowledgements}
I would like to express my sincere gratitude to all those who contributed to the successful completion of this thesis.
First and foremost, I extend my deepest appreciation to my thesis supervisor(s) [Name(s)] for their invaluable guidance, continuous support, and expertise throughout this research project.
Their insights into Machine Learning Develpment and practical feedback were instrumental in shaping the direction and quality of this work.
I am grateful to the team behind the LSFB (Langue des Signes de Belgique Francophone) project for providing access to their dataset.
Their collaboration was essential in demonstrating the real-world applicability of our MLOps-GitOps workflow, and their domain expertise greatly enriched the practical validation of our implementation.
Special thanks to my fellow students and peers who provided valuable feedback during the development process, particularly regarding the integration of GitOps methodology with our MLOps workflow.
Their suggestions and collaborative discussions helped refine our approach and improve the overall solution.
I would also like to acknowledge the open-source community and the developers of the various tools and frameworks that made this implementation possible, including Kubernetes, Docker, Helm, Airflow, and Kubeflow.
The extensive documentation and community support for these technologies were important resources throughout the development process.
Finally, I am grateful to my family and friends for their unwavering support and encouragement throughout my master's program and the completion of this thesis project.
This research would not have been possible without the collective contributions of all these individuals and communities.
