\section{Conclusion}\label{sec:stconclusion}

During our research and thanks to the survey we conducted we gain enough knowledge and confidence that our project was indeed of some usage.

As shown and stated in our results we can design an MLOps workflow thanks to already existing literature and tools.
But we didn't find any complete standard workflow that reunite Devops CI/CD pipelines and MLOps pipelines using traditional Git Branching strategies.
Tho we found GitOps cited in the literature about MLOps, we only find one articles with a proposed implementation to define a MLOps workflow following the GitOps principles\cite{mlops-gitops}.
We'll follow their remarks, and we'll try in our project to implement a MLOps/GitOps workflow using state-of-the-art frameworks and tools.

MLOps is stated to be the most efficient way to integrate ML models into production, with continuous training and mature systems leading to more realistic and effective models.\cite{inproceedings}

The integration of machine learning systems into production environments presents unique challenges that extend well beyond traditional software development practices.
As organizations increasingly rely on machine learning models to drive critical business decisions and user experiences,
the gap between experimental model development and robust production deployment has become a significant barrier to realizing value from artificial intelligence investments\cite{Haakman2021}.
This gap has given rise to MLOps—a discipline combining machine learning, DevOps, and data engineering practices to streamline the end-to-end machine learning lifecycle\cite{Kreuzberger2022MachineLO}.
Recent studies have highlighted the difficulties organizations face when transitioning machine learning from research to production.
