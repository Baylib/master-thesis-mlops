\section{Definitions}\label{sec:stdefinitions}
\subsection{DevOps}
We cannot approach MlOps without beforehand talking a bit about devops.
Devops is a software
Devops workflow: plan, code, build, test, release, deploy, operate and monitor.
MlOps should follow the devops principles, but it requires workflows to be extended as we will discuss later.
remark: DevSecOps

\subsubsection{GitOps} uses Git repository as the main source of truce.
It uses git sync and ci/cd pipelines to manage all operation from git repositories.
\subsection{CI/CD workflows}
\subsubsection{GitHubFlow}
\subsubsection{GitFlow}
\subsubsection{Trunk Based}
\subsection{Deployment Strategies}

To understand the details of a deployment pipeline, it is important to distinguish
among concepts often used inconsistently or interchangeably.
Integration
The process of merging a contribution to a central repository (typically merging
a Git feature branch to the main branch) and performing more or less complex
tests.
Delivery
As used in the continuous delivery (CD) part of CI/CD, the process of building a
fully packaged and validated version of the model ready to be deployed to
production.
Deployment
The process of running a new model version on a target infrastructure. Fully
automated deployment is not always practical or desirable and is a business deci‐
sion as much as a technical decision, whereas continuous delivery is a tool for the
development team to improve productivity and quality as well as measure pro‐
gress more reliably. Continuous delivery is required for continuous deployment,
but it also provides enormous value without.
Release
In principle, release is yet another step, as deploying a model version (even to the
production infrastructure) does not necessarily mean that the production work‐
load is directed to the new version. As we will see, multiple versions of a model
can run at the same time on the production infrastructure.\cite{treveil2020introducing}(p.77)
