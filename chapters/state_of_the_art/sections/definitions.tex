\section{Definitions}\label{sec:stdefinitions}
\subsection{DevOps}
We cannot approach MLOps without beforehand talking a bit about devops.
Devops is a software
Devops workflow: plan, code, build, test, release, deploy, operate and monitor.
MLOps should follow the devops principles, but it requires workflows to be extended as we will discuss later.
remark: DevSecOps

\subsubsection{GitOps}
GitOps is a DevOps approach that uses a Git repository as the single source of truth for system and application configurations.
It also provides change tracking through Git history and by storing infrastructure and configurations as code.
Tools like ArgoCD provide GitOps implementation for kubernetes\cite{gitops}.
One of the main aspect of GitOps is that it allows the cluster to pull changes directly from GitHub instead of a
traditional pushing to deploy.


\subsection{CI/CD workflows}
\subsubsection{GitHubFlow}
\subsubsection{GitFlow}
\subsubsection{Trunk Based}
\subsection{Deployment Strategies}

As explained in\cite{treveil2020introducing}(p.77), deployment strategies involve several key concepts.
Integration refers to merging code changes into a central repository and running tests,
while Delivery focuses on packaging and validating a model for production readiness.
Deployment is the process of running the new model version on target infrastructure,
which may or may not be fully automated, depending on business and technical considerations.
Continuous Delivery is essential for improving productivity and quality, even if full Continuous Deployment isn't implemented.
Finally, Release is distinct from deployment, as deploying a model doesn't always mean directing production traffic to it;
multiple model versions can coexist in production, allowing for controlled rollouts and testing.

