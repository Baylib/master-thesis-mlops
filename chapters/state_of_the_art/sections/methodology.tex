\section{Methodology}\label{sec:methodo}
\subsection{Literature review (LR)}\label{subsec:literature-review-(lr)}

We conducted a Literature Review in order to refine our research questions and find the key concepts and gaps in MLOps Workflows.
We proceeded as follows:

\begin{enumerate}
    \item Define our research questions
    \item Define articles inclusion and exclusion criteria
    \item Refine our set of keywords
    \item Querying publications search engines with our keywords
    \item Forward tracking and Back tracking
\end{enumerate}

We used this method because it is the \textit{de facto} approach to control biases and draws conclusions that stay partial due to
a non-complete coverage of the existing literature.

\subsection{Research questions}\label{subsec:research-questions}

We first defined several research questions and then refined the one that we deemed the most relevant.
Our first set of questions were:
\begin{itemize}
    \item What are the particularities of an MLOps workflow compared to a classic DevOps CI/CD workflow?
    \item What are the new requirements and challenges in MLOps?
    \item Are there/What are already mature tools to help development teams create pipelines for their ML projects
    \item What limitations of those tools can be identified?
\end{itemize}

Our final research question is
\textit{"How to implement/propose a state-of-the-art standard ML-GitOps workflow that could be used to gain maturity"\cite{mlops-definition-tools-and-challenge,mlops-maturity-model}}

\subsection{Articles relevance and inclusion/exclusion criteria}\label{subsec:articles-relevance-and-inclusion/exclusion-criteria}

Focus on MLOps with preference for python frameworks and architecture/product running on kubernetes.
Avoid Machine Learning or AI for DevOps articles.
Does not add any more information than already taken articles.
Discarded articles talking about hardware to avoid this aspect thx to kubernetes.
Only keep some documentation to receive advice on nodes hardware requirements.
Avoid cloud based / SaaS solutions.
also included some documentations about Devops, Gitops, Dataops to complete our definitions and general architecture.