\section{Methodology}\label{sec:methodo}
\subsection{Literature review (LR)}\label{subsec:literature-review-(lr)}

We conducted a Literature Review in order to refine our research questions and find the key concepts and gaps in MLOps Workflows.
We proceeded as follows:

\begin{enumerate}
    \item Define our research questions
    \item Define articles inclusion and exclusion criteria
    \item Refine our set of keywords
    \item Querying publications search engines with our keywords
    \item Forward tracking and Back tracking
\end{enumerate}

We used this method because it is the \textit{de facto} approach to control biases and draws conclusions that stay partial due to
a non-complete coverage of the existing literature.

\subsection{Research questions}\label{subsec:research-questions}

We first defined several research questions and then refined the one that we deemed the most relevant.
Our first set of questions were:
\begin{itemize}
    \item What are the particularities of an MLOps workflow compared to a classic DevOps CI/CD workflow?
    \item What are the new requirements and challenges in MLOps?
    \item Are there/What are already mature tools to help development teams create pipelines for their ML projects
    \item What limitations of those tools can be identified?
\end{itemize}

Our final research question is
\textit{"How can a state-of-the-art MLOps-GitOps workflow be implemented or proposed to initiate a machine learning project and support its progression toward greater maturity?"} %\cite{mlops-definition-tools-and-challenge,mlops-maturity-model}}

\subsection{Article Relevance and Inclusion/Exclusion Criteria}\label{subsec:articles-relevance-and-inclusion/exclusion-criteria}

The selection of articles was guided by the following criteria:
\begin{itemize}
    \item Focus on MLOps, with a preference for Python-based frameworks and architectures/products deployed on Kubernetes.
    \item Exclude articles about using Machine Learning or AI for DevOps (the reverse of MLOps).
    \item Exclude articles that do not provide additional insights beyond those already included.
    \item Discard articles centered on hardware-specific considerations, as Kubernetes abstracts away most hardware concerns.
    \item Retain only a few documentation sources related to hardware, specifically for understanding node requirements.
    \item Exclude cloud-based or SaaS-focused solutions.
    \item Include some documentation on related practices such as DevOps, GitOps, and DataOps to support broader definitions and architectural context.
\end{itemize}