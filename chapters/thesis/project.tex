\chapter{Thesis project: A standard MLOps CI/CD workflow}\label{ch:thesis-project:-a-standard-mlops-ci/cd-workflow}
\chaptermark{Thesis : MLOps WorkFlow}
\section{Introduction}\label{sec:introduction}
Following our state-of-the-art research we wanted to describe and implement a standard MLOps workflow.
I should be easily used on current project to improve their level of maturity or to initiate a new project with a solid
base to be fast in production.

\section{Tools}\label{sec:tools2}
In this section we will list all the tools we picked from our previous research.
As our workflow should be standard it should remain tools agnostic.

\subsection{Git Repositories and CI/CD}\label{subsec:github}
GitHub will be used to store and version control our codes repositories and configurations.
It will also be used to implement our CI/CD pipelines with GitHub actions.

\subsection{IAC, Containers and Registry}\label{subsec:dockerhub}
To benefit from containerization we will use docker to containerize our applications and models.
DockerHub will be used to store and version our Docker containers and Helm charts.

\subsection{Kubernetes}\label{subsec:kubernetes}
We will use a kubernetes cluster to benefit from its orchestration and deployment strategies.

\subsection{Airflow}\label{subsec:airflow}
We will use airflow for our DataOps pipelines and to integrate our kubeflow pipelines.

\subsection{Kubeflow}\label{subsec:kubeflow}
We will mainly use the storage solutions and pipelines offered by Kubeflow.

\subsection{ArgoCD}\label{subsec:argocd}
ArgoCD will be used to implement our GitOps strategy.

\section{Infrastructure}\label{sec:infrastructure}


\section{Workflow}\label{sec:workflow}


\section{Conclusion}\label{sec:conclusion}
Airflow pipelines and kubeflow pipelines to easily integrate DataOps and MLOps pipelines together to reach more automation as the project gains maturity.
