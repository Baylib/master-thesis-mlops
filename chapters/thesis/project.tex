\chapter{Thesis project: Implementing a MLOps/GitOps workflow}\label{ch:thesis-project:-a-standard-mlops-ci/cd-workflow}
\chaptermark{Thesis : MLOps/GitOps WorkFlow}
\section{Introduction}\label{sec:introduction}
Building upon our state-of-the-art review of MLOps practices, this chapter introduces our practical implementation of a MLOps/GitOps workflow.
The proposed workflow is designed with versatility in mind—capable of enhancing existing projects at various maturity levels or providing a robust foundation for new initiatives seeking accelerated production deployment.

Rather than developing yet another platform, our approach integrates established tools and workflows identified in the literature into a cohesive system tailored to address modern machine learning development challenges.
A distinguishing feature of our implementation is the incorporation of GitOps principles into the MLOps workflow, creating a synergistic relationship between these complementary methodologies.

At the core of our design is the GitOps philosophy, which establishes Git repositories as the single source of truth for code, configurations, and infrastructure.
We leverage GitHub repositories as the central foundation, with GitHub Actions orchestrating our DevOps workflows.
This integration extends further through tools such as ArgoCD, GitSync, and custom scripts that trigger our DataOps and MLOps pipelines. % in a synchronized manner.

Our infrastructure is deployable across one or multiple Kubernetes clusters, depending on specific project requirements.
This architecture decouples the workflow definition from hardware considerations, offering significant flexibility.
The Kubernetes implementation supports dedicated nodes with specialized resources (high CPU, memory, GPU) for compute-intensive machine learning or data processing operations.
Furthermore, this approach accommodates diverse deployment scenarios:

\begin{itemize}
\item Cloud-based Kubernetes clusters
\item On-premises infrastructure
\item Single-node Kubernetes configurations running on development laptops (within hardware constraints)
\end{itemize}

The seamless integration of DevOps, DataOps, and MLOps pipelines culminates in a comprehensive MLOps workflow that is portable across any Kubernetes environment.
This integration addresses the full machine learning lifecycle from development to deployment while maintaining consistent practices. % throughout.

Our implementation was specifically developed to support our promoter's ongoing LSFB (Langue des Signes de Belgique Francophone) project, which is already operational in a production environment.
The practical application of our workflow to an existing, real-world machine learning system provided valuable insights into the challenges of MLOps adoption beyond theoretical constructs.
By enhancing the established LSFB project infrastructure, we could demonstrate the adaptability and incremental benefits of our approach while supporting continued model development and improvement for this important initiative.

In the following sections, we detail our use cases and the specific tools selected to implement our workflow and explain how they interconnect to create a modular system that can adapt as projects evolve in complexity and maturity.
Notably, our tool selection was significantly influenced by the existing LSFB project infrastructure.
The current production system consists of a Python-based model deployed as a Docker image that exposes an API consumed by a web frontend.
The project already utilizes GitHub for code management and Airflow for certain pipeline operations, which represents partial implementation of practices identified in our literature review.
These existing elements served as foundational components that our comprehensive workflow needed to accommodate and enhance rather than replace, demonstrating the practical flexibility of our approach.

\section{Use Case Definitions}\label{sec:use-case-definitions}
Before proceeding further, we formally outline the use cases this project aims to address.
We will describe our use case diagram displayed in figure~\ref{fig:usecases}.

\begin{figure}[!htbp]
    \centering
    \caption{Example of a MLOps pipeline define within Airflow}
    \includegraphics[scale=0.5]{images/project/usecases}
    \label{fig:usecases}
\end{figure}

\subsection{Collaborative Model Development}\label{subsec:collaborative-model-development}
The platform must provide the necessary tools and infrastructure to enable development teams to collaboratively build,
test, and deploy models across segregated environments.

\subsection{Data Storage and Analysis}\label{subsec:data-storage-and-analysis}
A robust system for storing, processing, and analyzing data is required to support model training and evaluation.

\subsection{Model Versioning and Storage}\label{subsec:model-versioning-and-storage}
The solution must include a structured approach to versioning, storing, and retrieving trained models efficiently.

\subsection{Model and Tool Deployment}\label{subsec:model-and-tool-deployment}
The system must support seamless deployment of models and associated tooling across environments—from local development
setups to production—with minimal friction.

\subsection{Workflow Automation}\label{subsec:workflow-automation}
To enhance operational maturity, we prioritize automating repetitive tasks in the workflow.
This will be achieved through well-defined pipelines and triggers, informed by our state-of-the-art review.
\section{Tools Selection Rationale}\label{sec:tools2}
Having established the theoretical foundations and component descriptions in our state-of-the-art review, this section focuses specifically on our tool selection rationale.
While our workflow is designed to be fundamentally tool-agnostic, the following choices were made to address the specific requirements of our implementation context, particularly the existing LSFB project infrastructure.

\subsection{Python}\label{subsec:python}
The foundational programming language for our entire workflow implementation, selected primarily to maintain compatibility with the LSFB project's existing codebase and the broader machine learning ecosystem.
The project's current use of Keras for model development and our demonstration implementations using scikit-learn both benefit from Python's extensive machine learning library ecosystem and developer familiarity.
Notably, Python's pervasive role extends to our pipeline implementations, as both Airflow DAGs and Kubeflow pipelines are defined using Python, creating a consistent development environment across all workflow components.

\subsection{Git Repositories and CI/CD}\label{subsec:github}
We selected GitHub as our version control and CI/CD platform primarily due to its established presence in the LSFB project ecosystem.
This choice provides continuity with existing development practices while enabling us to leverage GitHub Actions for workflow automation without introducing additional integration complexity.
GitHub's robust API and extensive integration capabilities further support our goal of building an interconnected MLOps workflow.

\subsection{IAC, Containers and Registry}\label{subsec:dockerhub}
Docker was the natural containerization choice given the LSFB project's existing Docker-based model deployment.
DockerHub serves as our container registry and Helm chart repository, offering reliable accessibility and established integration paths with our other selected tools.
This approach maintains compatibility with the current production environment while enabling more sophisticated deployment patterns.

\subsection{Kubernetes}\label{subsec:kubernetes}
Kubernetes was selected as our orchestration platform for its exceptional flexibility and robust ecosystem.
Its ability to operate across various infrastructure environments (cloud, on-premises, development workstations) directly supports our portability requirements.
Furthermore, Kubernetes provides the foundation for advanced deployment strategies needed in ML systems and enables seamless integration with specialized tools like Kubeflow and Airflow.


\subsection{Airflow}\label{subsec:airflow}
We chose Airflow for DataOps orchestration to maintain compatibility with existing pipelines in the LSFB project.
This selection allows us to extend current functionality while providing a bridge to new MLOps capabilities.
Airflow's Python-based DAG definitions integrate naturally with our development workflow and allow data scientists to define complex pipelines using familiar syntax.
Airflow's ability to trigger Kubeflow pipelines creates a unified workflow that respects team boundaries and specialized tooling preferences while maintaining end-to-end process integrity.
Moreover, by leveraging Airflow's recent GitSync capabilities, we were able to successfully implement GitOps principles within our MLOps project.

\subsection{Kubeflow}\label{subsec:kubeflow}
Kubeflow was selected primarily for its storage solutions and pipeline capabilities that directly address ML-specific workflow requirements.
The Python SDK for Kubeflow Pipelines enables seamless integration with our existing Python codebase, allowing model developers to define reproducible ML workflows using familiar programming patterns.
This choice provides a growth path for the LSFB project, allowing incremental adoption of additional MLOps features as project maturity increases, without requiring architectural redesign.

\subsection{ArgoCD}\label{subsec:argocd}
ArgoCD was chosen as our GitOps implementation tool for its native Kubernetes integration and declarative approach to deployment automation.
This selection enables us to maintain infrastructure and application configurations as code within our GitHub repositories.
Additionally, Argo Rollouts provides sophisticated deployment capabilities essential for our model updates in any environments.

ArgoCD can be configured to trigger integration tests defined via helm test.
Its notification service may be used to send webhooks to GitHub upon successful deployment.
This contributes to a more fully automated workflow.

Additionally, ArgoCD’s project abstraction enables fine-grained, role-based access control (RBAC), allowing developers to deploy only predefined resources within authorized clusters and namespaces.
When combined with Kubernetes resource quotas, this approach empowers the infrastructure team to enforce secure and isolated development environments.

\section{Infrastructure}\label{sec:infrastructure}

In figure\ref{fig:project-infra}, we describe our infrastructure that can be first deployed on a local kubernetes single node cluster
with independent DataOps and MLOps pipelines develop by different people with different roles.

by using airflow pipelines as general pipelines to trigger our kubeflow pipelines, we can easily gain in maturity towards
more automation by combining the DataOps pipelines and MLOps pipelines together when they are mature enough.

This way we enable kubeflow features for our model developers within a more general purpose environment.

By using the recent git/sync capabilities of Airflow we can synchronise our dags directly with airflow and
even trigger them automatically in a later stage towards automation.

We use Git Repositories for version control and triggering workflow pipeline runners.
We consider 4 types of repositories:

\begin{itemize}
    \item Code Repositories that holds code for models, libraries, docker images and infrastructure as code using Helm.
    \item MLOps and DataOps Pipelines Repositories.
    In our infrastructure those repositories holds the definition of our Airflow DAGs.
    In the first iteration those repositories can also hold images for
    \item Deployment/Configuration repositories.
    Those repositories are used to hold configuration for the deployed applications and Airflow dags.
    We use ArgoCD GitOps implementation to synchronise changes to those repositories.
    Airflow git/sync feature allows us to synchronise our dags with
    Change in those repositories can be automated by the pipeline runners.
    Depending on your team and organization those repositories can be separated into multiple repositories (per team, per domain, per environment (dev,test,staging,prod))
    For the Dev environment we allow developers to push from their code repositories within those repositories.
    For the production environment we use a pipeline to pull the configuration from the staging environment.
    We called this promote our configuration to a new environment.
    In case of full automation the pulling/promotion can be trigger automatically by ArgoCD sending webhooks on any test results desired.
    \item Workflow pipeline templates
    In those we define workflows that can be used in any repository to be used by the developers.
    It includes building image workflows, versioning with tags on repositories, pushing new configurations into deployments repositories.
    By defining them in a separate repository it allows us to version them and make it easy for developers to choose a workflow.
    Each workflow is closely tight to the structure of the repository, so we defined one per type of repositories.
\end{itemize}

observability? monitoring and data observability with elasticserach?
more on metadata storage

\begin{figure}[!htbp]
    \centering
    \caption{Proposed MLOps kubernetes infrastructure using GitOps principles}
    \includegraphics[scale=0.35]{images/project/mthmlops-infra}
    \label{fig:project-infra}
\end{figure}

\section{Roles within our MLOps workflow}

Each team can be assigned and linked to our type of Git repositories define earlier\ref{subsec:git-repositories}.
All team can receive access to a Deployment/config repository that will be sync to the development environment.
DevOps and Operations teams should be responsible for the production repositories.

As we said earlier deployment repositories are synced to its defined environment but can be isolated using kubernetes namespaces,
and projects within Airflow, Kubeflow\footnote{require Kubeflow installed in multi-user mode} or ArgoCD.
This will ensure any team will deploy to their own namespaces with they predefined resource quotas.

\subsection{DevOps Engineers}
DevOps engineers are responsible for managing CI/CD pipeline templates, repositories, and the underlying runner infrastructure.
They should provide standardized templates for building and releasing Docker images, code libraries, and Helm charts.
Additionally, they should establish a process to synchronize configuration changes from a code repository to a deployment/configuration repository.
\subsection{Data Engineers}
The Data Team is responsible for maintaining at least one DAG repository to define their DataOps pipelines.
For more complex tasks, they should create dedicated code repositories to build custom images used within their DAGs.
\subsection{MLOps Engineers}
MLOps engineers are tasked with defining the overarching workflow and architecture, as demonstrated in this project.
They should oversee and manage relevant tools such as Airflow and Kubeflow, ensuring seamless integration and operation.
\subsection{Model Engineers}
Similar to the Data Team, Model Engineers should maintain at least one DAG repository to define their Kubeflow pipelines and build basic container images.
For more advanced steps, such as model and library development or model training, they should create separate repositories.
These repositories can follow a release workflow and be integrated as dependencies within the pipeline definitions.
\subsection{Software Engineers}
Software Engineers are responsible for developing the application that integrates and utilizes the model,
ensuring it meets functional requirements.
This application is defined as a Helm chart within our workflow, enabling deployment through ArgoCD.
For advanced deployment strategies, ArgoCD Rollouts can be utilized to ensure more refined and controlled release processes.
The trained model version is integrated into a REST API component, which is part of the application.

\subsection{Example of minimal list of git repositories per roles/teams within an Organisation}

Below, we present an example of the tree structure of an organization in GitHub.
It is divided into projects containing repositories.
Pipeline access between them is managed in GitHub using deployment tokens or SSH keys.

\begin{minted}{markdown}
Organisation
    -> DevOps Project
        -> CI/CD pipelines templates
    -> DataOps Project
        -> Airflow DAG definition with image for simple steps
        -> dataops-step-x
        -> dataops-step-y
    -> ML Project
        -> model code repository
        -> DAG/kubeflow pipeline repository
        -> mlops-pipeline-step-1
    -> ML Ops project
        -> Airflow/Kubeflow configuration
        -> infrastructure definition
    -> Operations
        -> production environment repositories subject to different rules.
\end{minted}

\section{Workflows}\label{sec:workflow}

\subsection{DevOps CI/CD pipelines}\label{subsec:ci/cd-pipelines-and-development-workflow}
As we explained while describing the infrastructure, part of our global workflow can be implemented within pipelines runners definition.
We used GitHub action and the GitHub flow to harmonise the workflow within our different type of git repositories.
GitHub allows us to define template in a single repository that can be then versioned and used in other repositories while hiding the complexity of the defined pipelines.
We also followed the GitHub flow which involve one main branch and features or hotfix branch to favor small and fast release.
As stated before other strategies can be used with an adapted development workflow.

\begin{figure}[!htbp]
    \centering
    \caption{Implementation of the GitHub Flow CI/CD workflow}
    \includegraphics[scale=0.3]{images/project/cicd-workflow-p1}
    \label{fig:icd-workflow-p1}
\end{figure}

There are our development pipelines for any git repositories except for the deployment repository that are only targets.
Artifact that are not deployed but released to a directory follows the same path but are deployed as dependencies within other projects.
Airflow DAGS follows the same development workflow and are release into our previously defined git deployment repositories.

\begin{figure}[!htbp]
    \centering
    \caption{Development Team Contribution activity}
    \includegraphics[scale=0.3]{images/project/cicd-workflow-p2}
    \label{fig:cd-workflow-p2}
\end{figure}


\subsection{DataOps pipelines}\label{subsec:dataops-pipelines}
Inspired by our research in the literature we create a sample DataOps pipeline that can be implemented with specific
operations.

\begin{figure}[!htbp]
    \centering
    \caption{DataOps sample workflow}
    \includegraphics[scale=0.3]{images/project/dataops-workflow}
    \label{fig:dataops-workflow}
\end{figure}

\begin{figure}[!htbp]
    \centering
    \caption{DataOps workflow define in Airflow}
    \includegraphics[scale=0.3]{images/project/dataops-workflow-airflow}
    \label{fig:dataops-workflow-airflow}
\end{figure}

It's defined in our project as an Airflow DAG and can be triggered manually in early stage of the project or
automatically when the project has enough maturity.

\subsection{MLOps pipelines}\label{subsec:mlops-pipelines}
Our MLOps pipelines are developed as Kubeflow pipelines with the possibility to trigger them from an airflow DAG
even in the early stage of the project in ensure the possibility to combined it with the DataOps pipeline to advance towards AutoML.

\begin{figure}[!htbp]
    \centering
    \caption{MLOps workflow in Kubeflow pipelines}
    \includegraphics[scale=0.3]{images/project/mlops-workflow-kubeflow}
    \label{fig:mlops-workflow-kubeflow}
\end{figure}

In our MLOps workflow, we have established a seamless integration between Apache Airflow and Kubeflow Pipelines to
efficiently manage the machine learning lifecycle.
This integration automates the process from data availability to model deployment, ensuring a streamlined and reproducible workflow.

\subsubsection{Data Availability Trigger}
AS explained in previous section, our DataOps pipeline preprocess and prepare the data to release train and test data's.
Airflow then triggers the Kubeflow pipeline to initiate the MLOps workflow.

\subsubsection{Model Training}
The pipeline begins with a training component that utilizes the provided datasets to train a machine learning model.
The trained model is saved to MinIO, our S3-compatible object storage integrated with Kubeflow.

\subsubsection{Model Validation}
After training, a validation component assesses the model's performance using the test data.
If the model meets our predefined accuracy thresholds and/or performs better than previous models,
the workflow proceeds; otherwise, it terminates, ensuring only validated models advance.

\subsubsection{Release and Deployment}
Upon successful validation, the pipeline enters the release phase, where the model and its metadata are saved to MinIO.
The deployment component updates our production environment with the new model, ensuring minimal disruption and continuous service delivery.


\section{Limitations and Critical Assessment}\label{sec:limitations-and-critical-assessment}

While our MLOps implementation demonstrates practical value, several aspects allows critical examination.
\subsection*{Architectural Complexity}
Our dual use of Kubeflow and Airflow may introduce unnecessary complexity and could potentially be simplified.
However, we deliberately maintained this separation for clear role distinction: Airflow handles data orchestration while Kubeflow manages ML-specific workflows.
We also found Airflow easier to manage multi-user and namespaces within kubernetes.
Future iterations could explore more integrated solutions, particularly as Airflow expands its MLOps capabilities.

\subsection*{Limited Production Validation}
Our implementation's real-world validation remains limited.
While we demonstrated practical utility through the LSFB dataset pipeline, we only validated the complete ML lifecycle using demonstration models rather than production-grade systems.
This restricts generalizability and leaves questions about performance.
\subsection*{Evolving Technology Landscape}
The rapidly evolving MLOps ecosystem, including developments like GitHub's beta model repository, may consolidate functionality that our implementation currently addresses through separate tools.
While our containerized approach provides adaptation flexibility, this field requires frequent architectural reassessment.

Despite these limitations, we maintain confidence in our implementation's core design principles and reusability through state-of-the-art practices including containerization, automated triggers, and well-defined parameters.


\section{Future Work}\label{sec:future-work}
This project is currently a prototype suitable for development and testing environments.
Future iterations should aim to harden the implementation for production use.
While the focus has primarily been on the MLOps workflow, several areas remain to be improved:

\begin{itemize}
    \item Production-Ready Infrastructure: The current setup lacks some of the considerations and robustness needed for a mature DevOps environment.
    \item Secret Management: Integrating a dedicated secrets manager such as HashiCorp Vault would enhance security and secret handling.
    \item Improved ArgoCD Usage: While we adopted the app-of-apps pattern in ArgoCD, enabling ApplicationSets could provide better self-service capabilities for development teams and simplify multi-environment management.
    \item Pipeline Orchestration Simplification: Currently, we use three different orchestrators.
    However, as platforms evolve—GitHub introducing a model registry, Kubeflow expanding its data management capabilities, and Airflow moving toward deeper MLOps integration—it may become feasible to consolidate to a single orchestrator in the future.
    \item Repository Naming: Standardizing and renaming Git repositories to clearer, more meaningful names will improve maintainability and team onboarding.
\end{itemize}

These improvements will help transition our prototype into a scalable, production-grade MLOps-GitOps solution.
% \url{https://argo-cd.readthedocs.io/en/stable/operator-manual/applicationset/Use-Cases/#use-case-self-service-of-argo-cd-applications-on-multitenant-clusters}.
%\url{https://argo-cd.readthedocs.io/en/stable/operator-manual/secret-management/}.
\section{Conclusion}\label{sec:conclusion}

During this project, we were able to implement the techniques and concepts learned throughout our master's program, specializing in areas such as Kubernetes clusters, DevOps, DataOps, MLOps, Docker, Helm, and Python programming.
Through literature research and reviewing various documentation, we successfully implemented an MLOps workflow prototype capable of supporting enterprise-level infrastructure and being reused across multiple data engineering and machine learning projects.
We also explored integrating a GitOps methodology with our MLOps workflow, following feedback from our peers.

We answered our research question \textit{"How can a state-of-the-art MLOps-GitOps workflow be implemented or proposed to initiate a machine learning project and support its progression toward greater maturity?"}
, by implementing a state-of-the-art MLOps-GitOps workflow that can be use as a project bootstrap and gain maturity over time.

As the project matures, we integrated Airflow and Kubeflow pipelines to enable seamless automation of both our DataOps and MLOps workflows.
By applying the same GitOps methodology for managing our DataOps and MLOps pipelines as well as for deploying infrastructure and applications via HelmCharts,
we have successfully established a consistent and unified workflow for all development processes, whether for application deployment, Airflow DAG orchestration or Kubeflow pipelines orchestration.

The LSFB dataset provided by our promoters was instrumental in demonstrating the applicability of our workflow, using a real model already in production.
Their expertise offered valuable insights and served as a strong foundation for initiating our project.

We integrated all necessary elements for our implementation in accordance with the findings from our state-of-the-art review.
To conclude, we have summarized our results in the table below.

%tableau avec infos ur notre implémentation.
%\begin{landscape}
\footnotesize
\begin{longtable}{|p{5cm}|p{4cm}|p{8cm}|}
    \hline
    \textbf{Good Practice} & \textbf{Tool Used} & \textbf{Remarks}\\
    \hline
    CI/CD automation & GitHub Actions & We implemented the GitOps principles into our DevOps pipelines \\
    \hline
    Pipeline/Workflow orchestration & Kubeflow/Airflow Pipelines & We chose a container based approach to ease reusability \\
    \hline
    Reproducibility using pipeline orchestrator & Kubeflow, Airflow &  Both Airflow and Kubeflow DSL ease the experiment tracking and reproducibility \\
    \hline
    Versioning of data, code and model with model registry, code repositories and container registries & GitHub, Docker Hub & Makes it easy to create free account for student usage\\
    \hline
    Collaboration with collaborative development platform & GitHub & We specified rules and branch strategy to manage our repositories \\
    \hline
    Continuous ML training and evaluation with well-defined triggers for pipelines & Apache Airflow, Kubeflow Pipelines & We implemented and demonstrated use of configurable pipelines with clearly defined input/output parameters\\
    \hline
    ML metadata tracking & Kubeflow & We used Kubeflow DSL to ease metadata management\\
    \hline
    Continuous Monitoring (Monitoring system) & ElasticSearch & Only tested in early stages of the project\\
    \hline
    Feedback loops and Data Drift Detectors, Human Verifications & ElasticSearch & Only tested in early stages of the project\\
    \hline
\end{longtable}
%\end{landscape}
\normalsize

