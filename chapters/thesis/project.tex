\chapter{Thesis project: A standard MLOps/GitOps workflow}\label{ch:thesis-project:-a-standard-mlops-ci/cd-workflow}
\chaptermark{Thesis : MLOps/GitOps WorkFlow}
\section{Introduction}\label{sec:introduction}
Following our state-of-the-art research we wanted to describe and implement a standard MLOps workflow.
I should be easily used on current project to improve their level of maturity or to initiate a new project with a solid
base to be fast in production.

We don't want to create a platform, but we want to use and integrate together reviewed
tools and workflows we encountered in the literature to propose our own.

In this project, we want to apply the GitOps principle to our MLOps workflow in order to combine those two ideas together.
GitOps uses Git repositories as the main source of truce for the code, the configurations and the infrastructure.
In this matter we defined all the project within GitHub repositories to use GitHub Actions to trigger our DevOps workflow.
With the help of tools like ArgoCD, GitSync or custom scripts we also manage to trigger our DataOps and MLOps pipelines.

All our infrastructure can be deployed within one or more kubernetes clusters, depending on the actual requirements of
the ML project, this allows to define a workflow independently of hardware consideration.
Dedicated nodes with special requirements (high CPU, memory, GPU) for machine learning or data operations can be define
to run dedicated steps.
A kubernetes cluster can be deployed on premise or within the cloud depending on current infrastructure.
A single node kubernetes cluster running on the engineer laptop can be used as a development environment
within its hardware capabilities.

By combining DevOps, DataOps and MLOps pipelines with get a full MLOps workflow that can be easily deployed on any kubernetes clusters.

We will now briefly describe the tools we chose to implement our workflow squeleton.

\section{Tools}\label{sec:tools2}
In this section we will list all the tools we picked from our previous research.
As our workflow should be standard it should remain tools agnostic.

\subsection{Git Repositories and CI/CD}\label{subsec:github}
GitHub will be used to store and version control our codes repositories and configurations.
It will also be used to implement our CI/CD pipelines with GitHub actions.

\subsection{IAC, Containers and Registry}\label{subsec:dockerhub}
To benefit from containerization we will use docker to containerize our applications and models.
DockerHub will be used to store and version our Docker containers and Helm charts.

\subsection{Kubernetes}\label{subsec:kubernetes}
We will use a Kubernetes cluster to take advantage of its orchestration capabilities, advanced deployment strategies,
and the ecosystem of tools available on the platform.

\subsection{Airflow}\label{subsec:airflow}
We will use Airflow for our DataOps pipelines and to integrate our kubeflow pipelines.
We can use an Airflow DAG to trigger our Kubeflow pipelines and have a complete DataOps/MLOps pipeline while still allow respective teams
to benefit from other tools.

\subsection{Kubeflow}\label{subsec:kubeflow}
We will mainly use the storage solutions and pipelines offered by Kubeflow.
But by using kubeflow pipelines we could easily add features during the development if required by the Data or Model engineers.

\subsection{ArgoCD}\label{subsec:argocd}
ArgoCD will be used to implement our GitOps strategy and deploy our application, models, airflow and kubeflow.
We used Argo rollouts for canary deployment strategies.

\section{Infrastructure}\label{sec:infrastructure}

In figure\ref{fig:project-infra}, we describe our infrastructure that can be first deployed on a local kubernetes single node cluster
with independent DataOps and MLOps pipelines develop by different people with different roles.

by using airflow pipelines as general pipelines to trigger our kubeflow pipelines, we can easily gain in maturity towards
more automation by combining the DataOps pipelines and MLOps pipelines together when they are mature enough.

This way we enable kubeflow features for our model developers within a more general purpose environment.

By using the recent git/sync capabilities of Airflow we can synchronise our dags directly with airflow and
even trigger them automatically in a later stage towards automation.

We use Git Repositories for version control and triggering workflow pipeline runners.
We consider 4 types of repositories:

\begin{itemize}
    \item Code Repositories that holds code for models, libraries, docker images and infrastructure as code using Helm.
    \item MLOps and DataOps Pipelines Repositories.
    In our infrastructure those repositories holds the definition of our Airflow DAGs.
    In the first iteration those repositories can also hold images for
    \item Deployment/Configuration repositories.
    Those repositories are used to hold configuration for the deployed applications and Airflow dags.
    We use ArgoCD GitOps implementation to synchronise changes to those repositories.
    Airflow git/sync feature allows us to synchronise our dags with
    Change in those repositories can be automated by the pipeline runners.
    Depending on your team and organization those repositories can be separated into multiple repositories (per team, per domain, per environment (dev,test,staging,prod))
    For the Dev environment we allow developers to push from their code repositories within those repositories.
    For the production environment we use a pipeline to pull the configuration from the staging environment.
    We called this promote our configuration to a new environment.
    In case of full automation the pulling/promotion can be trigger automatically by ArgoCD sending webhooks on any test results desired.
    \item Workflow pipeline templates
    In those we define workflows that can be used in any repository to be used by the developers.
    It includes building image workflows, versioning with tags on repositories, pushing new configurations into deployments repositories.
    By defining them in a separate repository it allows us to version them and make it easy for developers to choose a workflow.
    Each workflow is closely tight to the structure of the repository, so we defined one per type of repositories.
\end{itemize}

observability? monitoring and data observability with elasticserach?
more on metadata storage

\begin{figure}[!htbp]
    \centering
    \caption{Proposed MLOps kubernetes infrastructure using GitOps principles}
    \includegraphics[scale=0.35]{images/project/mthmlops-infra}
    \label{fig:project-infra}
\end{figure}

\section{Roles within our MLOps workflow}

Each team can be assigned and linked to our type of Git repositories define earlier\ref{subsec:git-repositories}.
All team can receive access to a Deployment/config repository that will be sync to the development environment.
DevOps and Operations teams should be responsible for the production repositories.

As we said earlier deployment repositories are synced to its defined environment but can be isolated using kubernetes namespaces,
and projects within Airflow, Kubeflow\footnote{require Kubeflow installed in multi-user mode} or ArgoCD.
This will ensure any team will deploy to their own namespaces with they predefined resource quotas.

\subsection{DevOps Engineers}
DevOps engineers are responsible for managing CI/CD pipeline templates, repositories, and the underlying runner infrastructure.
They should provide standardized templates for building and releasing Docker images, code libraries, and Helm charts.
Additionally, they should establish a process to synchronize configuration changes from a code repository to a deployment/configuration repository.
\subsection{Data Engineers}
The Data Team is responsible for maintaining at least one DAG repository to define their DataOps pipelines.
For more complex tasks, they should create dedicated code repositories to build custom images used within their DAGs.
\subsection{MLOps Engineers}
MLOps engineers are tasked with defining the overarching workflow and architecture, as demonstrated in this project.
They should oversee and manage relevant tools such as Airflow and Kubeflow, ensuring seamless integration and operation.
\subsection{Model Engineers}
Similar to the Data Team, Model Engineers should maintain at least one DAG repository to define their Kubeflow pipelines and build basic container images.
For more advanced steps, such as model and library development or model training, they should create separate repositories.
These repositories can follow a release workflow and be integrated as dependencies within the pipeline definitions.
\subsection{Software Engineers}
Software Engineers are responsible for developing the application that integrates and utilizes the model,
ensuring it meets functional requirements.
This application is defined as a Helm chart within our workflow, enabling deployment through ArgoCD.
For advanced deployment strategies, ArgoCD Rollouts can be utilized to ensure more refined and controlled release processes.
The trained model version is integrated into a REST API component, which is part of the application.

\subsection{Example of minimal list of git repositories per roles/teams within an Organisation}

Below, we present an example of the tree structure of an organization in GitHub.
It is divided into projects containing repositories.
Pipeline access between them is managed in GitHub using deployment tokens or SSH keys.

\begin{minted}{markdown}
Organisation
    -> DevOps Project
        -> CI/CD pipelines templates
    -> DataOps Project
        -> Airflow DAG definition with image for simple steps
        -> dataops-step-x
        -> dataops-step-y
    -> ML Project
        -> model code repository
        -> DAG/kubeflow pipeline repository
        -> mlops-pipeline-step-1
    -> ML Ops project
        -> Airflow/Kubeflow configuration
        -> infrastructure definition
    -> Operations
        -> production environment repositories subject to different rules.
\end{minted}

\section{Workflows}\label{sec:workflow}

\subsection{DevOps CI/CD pipelines}\label{subsec:ci/cd-pipelines-and-development-workflow}
As we explained while describing the infrastructure, part of our global workflow can be implemented within pipelines runners definition.
We used GitHub action and the GitHub flow to harmonise the workflow within our different type of git repositories.
GitHub allows us to define template in a single repository that can be then versioned and used in other repositories while hiding the complexity of the defined pipelines.
We also followed the GitHub flow which involve one main branch and features or hotfix branch to favor small and fast release.
As stated before other strategies can be used with an adapted development workflow.

\begin{figure}[!htbp]
    \centering
    \caption{Implementation of the GitHub Flow CI/CD workflow}
    \includegraphics[scale=0.3]{images/project/cicd-workflow-p1}
    \label{fig:icd-workflow-p1}
\end{figure}

There are our development pipelines for any git repositories except for the deployment repository that are only targets.
Artifact that are not deployed but released to a directory follows the same path but are deployed as dependencies within other projects.
Airflow DAGS follows the same development workflow and are release into our previously defined git deployment repositories.

\begin{figure}[!htbp]
    \centering
    \caption{Development Team Contribution activity}
    \includegraphics[scale=0.3]{images/project/cicd-workflow-p2}
    \label{fig:cd-workflow-p2}
\end{figure}


\subsection{DataOps pipelines}\label{subsec:dataops-pipelines}
Inspired by our research in the literature we create a sample DataOps pipeline that can be implemented with specific
operations.

\begin{figure}[!htbp]
    \centering
    \caption{DataOps sample workflow}
    \includegraphics[scale=0.3]{images/project/dataops-workflow}
    \label{fig:dataops-workflow}
\end{figure}

\begin{figure}[!htbp]
    \centering
    \caption{DataOps workflow define in Airflow}
    \includegraphics[scale=0.3]{images/project/dataops-workflow-airflow}
    \label{fig:dataops-workflow-airflow}
\end{figure}

It's defined in our project as an Airflow DAG and can be triggered manually in early stage of the project or
automatically when the project has enough maturity.

\subsection{MLOps pipelines}\label{subsec:mlops-pipelines}
Our MLOps pipelines are developed as Kubeflow pipelines with the possibility to trigger them from an airflow DAG
even in the early stage of the project in ensure the possibility to combined it with the DataOps pipeline to advance towards AutoML.

\begin{figure}[!htbp]
    \centering
    \caption{MLOps workflow in Kubeflow pipelines}
    \includegraphics[scale=0.3]{images/project/mlops-workflow-kubeflow}
    \label{fig:mlops-workflow-kubeflow}
\end{figure}

In our MLOps workflow, we have established a seamless integration between Apache Airflow and Kubeflow Pipelines to
efficiently manage the machine learning lifecycle.
This integration automates the process from data availability to model deployment, ensuring a streamlined and reproducible workflow.

\subsubsection{Data Availability Trigger}
AS explained in previous section, our DataOps pipeline preprocess and prepare the data to release train and test data's.
Airflow then triggers the Kubeflow pipeline to initiate the MLOps workflow.

\subsubsection{Model Training}
The pipeline begins with a training component that utilizes the provided datasets to train a machine learning model.
The trained model is saved to MinIO, our S3-compatible object storage integrated with Kubeflow.

\subsubsection{Model Validation}
After training, a validation component assesses the model's performance using the test data.
If the model meets our predefined accuracy thresholds and/or performs better than previous models,
the workflow proceeds; otherwise, it terminates, ensuring only validated models advance.

\subsubsection{Release and Deployment}
Upon successful validation, the pipeline enters the release phase, where the model and its metadata are saved to MinIO.
The deployment component updates our production environment with the new model, ensuring minimal disruption and continuous service delivery.




\section{Conclusion}\label{sec:conclusion}
Airflow pipelines and kubeflow pipelines to easily integrate DataOps and MLOps pipelines together to reach more automation as the project gains maturity.
