\section{Limitations and Critical Assessment}\label{sec:limitations-and-critical-assessment}

While our MLOps implementation demonstrates practical value, several aspects allows critical examination.
\subsection*{Architectural Complexity}
Our dual use of Kubeflow and Airflow may introduce unnecessary complexity and could potentially be simplified.
However, we deliberately maintained this separation for clear role distinction: Airflow handles data orchestration while Kubeflow manages ML-specific workflows.
We also found Airflow easier to manage multi-user and namespaces within kubernetes.
Future iterations could explore more integrated solutions, particularly as Airflow expands its MLOps capabilities.

\subsection*{Limited Production Validation}
Our implementation's real-world validation remains limited.
While we demonstrated practical utility through the LSFB dataset pipeline, we only validated the complete ML lifecycle using demonstration models rather than production-grade systems.
This restricts generalizability and leaves questions about performance.
\subsection*{Evolving Technology Landscape}
The rapidly evolving MLOps ecosystem, including developments like GitHub's beta model repository, may consolidate functionality that our implementation currently addresses through separate tools.
While our containerized approach provides adaptation flexibility, this field requires frequent architectural reassessment.

Despite these limitations, we maintain confidence in our implementation's core design principles and reusability through state-of-the-art practices including containerization, automated triggers, and well-defined parameters.

