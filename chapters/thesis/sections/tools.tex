\section{Tools Selection Rationale}\label{sec:tools2}
Having established the theoretical foundations and component descriptions in our state-of-the-art review, this section focuses specifically on our tool selection rationale.
While our workflow is designed to be fundamentally tool-agnostic, the following choices were made to address the specific requirements of our implementation context, particularly the existing LSFB project infrastructure.

\subsection{Python}\label{subsec:python}
The foundational programming language for our entire workflow implementation, selected primarily to maintain compatibility with the LSFB project's existing codebase and the broader machine learning ecosystem.
The project's current use of Keras for model development and our demonstration implementations using scikit-learn both benefit from Python's extensive machine learning library ecosystem and developer familiarity.
Notably, Python's pervasive role extends to our pipeline implementations, as both Airflow DAGs and Kubeflow pipelines are defined using Python, creating a consistent development environment across all workflow components.

\subsection{Git Repositories and CI/CD}\label{subsec:github}
We selected GitHub as our version control and CI/CD platform primarily due to its established presence in the LSFB project ecosystem.
This choice provides continuity with existing development practices while enabling us to leverage GitHub Actions for workflow automation without introducing additional integration complexity.
GitHub's robust API and extensive integration capabilities further support our goal of building an interconnected MLOps workflow.

\subsection{IAC, Containers and Registry}\label{subsec:dockerhub}
Docker was the natural containerization choice given the LSFB project's existing Docker-based model deployment.
DockerHub serves as our container registry and Helm chart repository, offering reliable accessibility and established integration paths with our other selected tools.
This approach maintains compatibility with the current production environment while enabling more sophisticated deployment patterns.

\subsection{Kubernetes}\label{subsec:kubernetes}
Kubernetes was selected as our orchestration platform for its exceptional flexibility and robust ecosystem.
Its ability to operate across various infrastructure environments (cloud, on-premises, development workstations) directly supports our portability requirements.
Furthermore, Kubernetes provides the foundation for advanced deployment strategies needed in ML systems and enables seamless integration with specialized tools like Kubeflow and Airflow.


\subsection{Airflow}\label{subsec:airflow}
We chose Airflow for DataOps orchestration to maintain compatibility with existing pipelines in the LSFB project.
This selection allows us to extend current functionality while providing a bridge to new MLOps capabilities.
Airflow's Python-based DAG definitions integrate naturally with our development workflow and allow data scientists to define complex pipelines using familiar syntax.
Airflow's ability to trigger Kubeflow pipelines creates a unified workflow that respects team boundaries and specialized tooling preferences while maintaining end-to-end process integrity.
Moreover, by leveraging Airflow's recent GitSync capabilities, we were able to successfully implement GitOps principles within our MLOps project.

\subsection{Kubeflow}\label{subsec:kubeflow}
Kubeflow was selected primarily for its storage solutions and pipeline capabilities that directly address ML-specific workflow requirements.
The Python SDK for Kubeflow Pipelines enables seamless integration with our existing Python codebase, allowing model developers to define reproducible ML workflows using familiar programming patterns.
This choice provides a growth path for the LSFB project, allowing incremental adoption of additional MLOps features as project maturity increases, without requiring architectural redesign.

\subsection{ArgoCD}\label{subsec:argocd}
ArgoCD was chosen as our GitOps implementation tool for its native Kubernetes integration and declarative approach to deployment automation.
This selection enables us to maintain infrastructure and application configurations as code within our GitHub repositories.
Additionally, Argo Rollouts provides sophisticated deployment capabilities essential for our model updates in any environments.

ArgoCD can be configured to trigger integration tests defined via helm test.
Its notification service may be used to send webhooks to GitHub upon successful deployment.
This contributes to a more fully automated workflow.

Additionally, ArgoCD’s project abstraction enables fine-grained, role-based access control (RBAC), allowing developers to deploy only predefined resources within authorized clusters and namespaces.
When combined with Kubernetes resource quotas, this approach empowers the infrastructure team to enforce secure and isolated development environments.
