\section{Roles within our MLOps workflow}

Each team can be assigned and linked to our type of Git repositories define earlier\ref{subsec:git-repositories}.
All team can receive access to a Deployment/config repository that will be sync to the development environment.
DevOps and Operations teams should be responsible for the production repositories.

As we said earlier deployment repositories are synced to its defined environment but can be isolated using kubernetes namespaces,
and projects within Airflow, Kubeflow\footnote{require Kubeflow installed in multi-user mode} or ArgoCD.
This will ensure any team will deploy to their own namespaces with they predefined resource quotas.

\subsection{DevOps Engineers}
DevOps engineers are responsible for managing CI/CD pipeline templates, repositories, and the underlying runner infrastructure.
They should provide standardized templates for building and releasing Docker images, code libraries, and Helm charts.
Additionally, they should establish a process to synchronize configuration changes from a code repository to a deployment/configuration repository.
\subsection{Data Engineers}
The Data Team is responsible for maintaining at least one DAG repository to define their DataOps pipelines.
For more complex tasks, they should create dedicated code repositories to build custom images used within their DAGs.
\subsection{MLOps Engineers}
MLOps engineers are tasked with defining the overarching workflow and architecture, as demonstrated in this project.
They should oversee and manage relevant tools such as Airflow and Kubeflow, ensuring seamless integration and operation.
\subsection{Model Engineers}
Similar to the Data Team, Model Engineers should maintain at least one DAG repository to define their Kubeflow pipelines and build basic container images.
For more advanced steps, such as model and library development or model training, they should create separate repositories.
These repositories can follow a release workflow and be integrated as dependencies within the pipeline definitions.
\subsection{Software Engineers}
Software Engineers are responsible for developing the application that integrates and utilizes the model,
ensuring it meets functional requirements.
This application is defined as a Helm chart within our workflow, enabling deployment through ArgoCD.
For advanced deployment strategies, ArgoCD Rollouts can be utilized to ensure more refined and controlled release processes.
The trained model version is integrated into a REST API component, which is part of the application.

\subsection{Example of minimal list of git repositories per roles/teams within an Organisation}

Below, we present an example of the tree structure of an organization in GitHub.
It is divided into projects containing repositories.
Pipeline access between them is managed in GitHub using deployment tokens or SSH keys.

\begin{minted}{markdown}
Organisation
    -> DevOps Project
        -> CI/CD pipelines templates
    -> DataOps Project
        -> Airflow DAG definition with image for simple steps
        -> dataops-step-x
        -> dataops-step-y
    -> ML Project
        -> model code repository
        -> DAG/kubeflow pipeline repository
        -> mlops-pipeline-step-1
    -> ML Ops project
        -> Airflow/Kubeflow configuration
        -> infrastructure definition
    -> Operations
        -> production environment repositories subject to different rules.
\end{minted}
